\documentclass[10pt,letterpaper]{article}
\usepackage[latin1]{inputenc}
\usepackage{amsmath}
\usepackage{amsfonts}
\usepackage{amssymb}
\usepackage{graphicx}


\author{Andrew Rosen}
\title{MBD Talk 2 Writeup}
\begin{document}
	\maketitle
	
	
	This talk focused on a compound that inhibited g-aminobutyric acid aminotransferase (GABA-AT) \cite{gata}.
	Much of the focus is on GABA-AT inhibition directed at two applications:  epilepsy and drug use.
	
	Epilepsy is a very old condition, which is loosely defined as  any central nervous system disorder which presents recurring seizures.
	It affects approximated 1-2\% of the human population, and 30-40\% of those afflicted have no effective treatment options.
	One of the correlations of convulsions in patients is a decrease in GABA levels; so if we can increase GABA levels in patients, we can decrease incidences of convulsions.
	This requires a molecule that can perform two tasks: cross the blood-brain barrier and inhibit GABA-AT.
	
	One drug that can do this is Vigabatrin, but it is hindered by needing a large dose (1-3 grams daily) to achieve a noticeable effect.
	A serious side-effect of the drug is it causes retinal damage via an unknown mechanism.
	
	The lecturer presented a molecule, (1s,3s)-3-amino-4-difluoromethylenyl-1-cyclopentanoic acid (CPP-115), that he had a hand in discovering \cite{silverman2011methods}.  
	Glossing over the chemical details which were the bulk of the lecture, CPP-115 is able to inhibit GABA-AT, without the retinal damage or extremely large dosage.
	It is many more times potent then Vigabatrin.
	
	
	GABA-AT inibition is also valuble in treating addiction drugs.
	Again, simplifying chemical details, addicting substances are characterized by an increase in dopamine.
	The dopamine increase can be antagonized by an increase in GABA.
	CPP-115 effectively inhibited the increase of dopamine  in lab rats\cite{pan20111}.
	Lab rabs were given the choice between an environment with cocaine, and an enviroment without, after being forced into both for a period of time.
	Rats given CPP-115 essentially choose at random.
	
	
	\bibliographystyle{plain}
	\bibliography{mbd}
\end{document}