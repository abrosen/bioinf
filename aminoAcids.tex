\documentclass[10pt,letterpaper]{article}
\usepackage[latin1]{inputenc}
\usepackage{amsmath}
\usepackage{amsfonts}
\usepackage{amssymb}
\usepackage{graphicx}


\author{Andrew Rosen}
\title{Amino Acid Writeup}
\begin{document}
\maketitle
%Study the two amino acids with different properties listed by your name below.
%Write 1-2 pages (total) in your own words on the physical and chemical properties of the 2 amino acids, and their importance for protein structure. 
%A picture of the chemical structure of the amino acids can be included. 

\section{Aspartic Acid}
Aspartic Acid (standard codons \texttt{GAC} and \texttt{GAU}) \cite{betts2003amino} is a non-essential amino acid \cite{young1994adult}.\footnote{A highly misleading attribute name.}
Its chemical formula is $\mathtt{ HOOCCH(NH_2)CH_2COOH} $.
It is polar, which means it is hydrophilic \cite{petitjean2007impact} and generally ends up on the surface of proteins.
Aspartic Acid is small compared to other amino acids.
It is usually found in the protein active sites.

It is one of two negatively charged amino acids, so its use in protein structure stems from the ability of the carboxylate ($ \mathtt{RCOO^{-}})$ to bind with positively charged amino acids.
This particular binding forms hydrogen bonds and keeps the protein's structure stable.
The negative charge also allows Aspartic acid to interact with positive atoms.



\section{Phenylalanine}
Phenylalanine  (standard codons \texttt{UUC} and \texttt{UUU}) \cite{betts2003amino} is a hydrophobic, non-polar amino acid \cite{petitjean2007impact}. 
Phenylalanine is an essential amino acid and cannot be synthesized \cite{young1994adult}.
Phenylalanine generally ends up in the hydrophobic core of the protein since it itself is hydrophobic.


The chemical formula for Phenylalanine is $\mathtt{C_{6}H_{5}CH_{2}CH(NH_{2})COOH}$ and has a non-reactive benzyl side chain.
This means Phenylalanine does not generally impact protein function, but since the side chain is aromatic (meaning the electrons of the ring are shared with the whole ring), it will stack with other aromatic amino acids such as Tryptophan.


\bibliography{bio}
\bibliographystyle{acm}
\end{document}