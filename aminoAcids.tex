\documentclass[10pt,letterpaper]{article}
\usepackage[latin1]{inputenc}
\usepackage{amsmath}
\usepackage{amsfonts}
\usepackage{amssymb}
\usepackage{graphicx}


\author{Andrew Rosen}
\title{Amino Acid Writeup}
\begin{document}
\maketitle
%Study the two amino acids with different properties listed by your name below.
%Write 1-2 pages (total) in your own words on the physical and chemical properties of the 2 amino acids, and their importance for protein structure. 
%A picture of the chemical structure of the amino acids can be included. 

\section{Aspartic Acid}
Aspartic Acid (standar codons \texttt{GAC} and \texttt{GAU}) is a non-essential amino acid \cite{young1994adult}.\footnote{A highly misleading attribute name.}
It is polar, which means it is hydrophopbic and generally ends up on the surface of proteins.
Aspartic Acid is small compared to other amino acids.

It is one of two negatively charged amino acids, so its use in protein structure stems from the ability of the carbolyxate ($ \mathtt{RCOO^{-}})$ to bind with positively charged amino acids.
This particular binding forms hydrogen bonds and keeps the protein's structure stable.

\texttt{GAC} and \texttt{GAU}


\section{Phenylalanine}
The chemical formula for Phenylalaine is $ \mathtt{C_9H_{11}NO_2}$.
stardard codons are \texttt{UUC} and \texttt{UUU} \cite{betts2003amino}.
It is extremely hydrophobic, with 


is an essential amino acid \cite{young1994adult}, meaning it is not synthersized internally and must be consumed.

\bibliography{bio}
\bibliographystyle{acm}
\end{document}