% !TeX spellcheck = en_US
\documentclass[10pt,letterpaper]{article}
\usepackage[latin1]{inputenc}
\usepackage{amsmath}
\usepackage{amsfonts}
\usepackage{amssymb}
\usepackage{graphicx}
\author{Andrew Rosen}
\title{Group Project Report: siRNA Generator}
\date{}
\begin{document}

\maketitle

\section{Overview}
I will first discuss the motivation for our work and our group's broad objective.


\subsection{Motivation}
Our primary motivation for our project was the 2014 Ebola outbreak.
%which species 
There are five identified species of the Ebolavirus: Bundibugyo, Ta\"{\i}, Sudan, and Zaire \cite{centers20152014}.
The last, the Reston virus, does not harm humans.
The Zaire ebolavirus is the most deadly and the one responsible for the current outbreak, the most deadly to date.

%symptionm
Symptoms of Ebola virus disease (EVD) include: fever, chills, fatigue, weakness, muscle/join pain,  vomiting, reddened eyes, and hemorrhaging \cite{team2014ebola} \cite{wongcharacterization}.
These symptoms appear within eight to ten days, on average \cite{centers20152014}.


This outbreak primarily affected people in Guinea, Liberia, and Sierra Leone, with additional cases reported in Nigeria, Mali, and Senegal \cite{centers20152014} .
Thus far, there have been an estimated 25,907 infected in 2014 outbreak, and nearly 11,000 people have succumbed to the disease \cite{centers20152014}.
Two cases of Ebola were imported to the United States and was spread to two nurses. 
This, plus cases Spain and the United Kingdom sparked a large level of concern worldwide  \cite{levin2015ebola} \cite{ready}.


% prevention
Ebola can be fought, but the most effective method is prevention.
Ebola does not spread easily when compared to other viruses such as influenza.
Infection requires contact with the bodily fluids of a infected person who is displaying symptoms of Ebola virus disease \cite{wongcharacterization} \cite{team2014ebola}.
Thus, the most effective way currently available is to quarantine the infected individuals and inform those who have had contact with the infected \cite{team2014ebola}.
% suspicion, 

\subsection{Our approach}
We chose using siRNA since a couple of group members were involved in active research on siRNA and were very familiar with it.

shRNA is precursor to siRNA
Binds to mRNA in virus in order to render it ineffective.


%Generalize it.
\section{Contributions}




\subsection{Assigned Tasks}
Our group separated into three rough groups to tackle each of the three tasks we identified.

The first group was tasked with creating a script to identify the conserved region within the species of virus.
These conserved regions would then be compared against the human genome using BLAST \cite{blast}.
Any sequence that was shared with the human genome would be filtered out of output for the next group, since we targt that sequence with an siRNA, it could potentially target that sequence in humans.\footnote{This is generally considered a bad idea}

The second group takes the conserved regions that did not show up in the human genome and uses those to design siRNA to bind with the virus's mRNA.
The resulting siRNA needs to then be compared against the human genome, for the same reasons as the first check

I was the third group and my responsibilities were quite broad.
My responsibilities included helping set up the server we would use for our application.


Creating and maintaining server
Obtaining tools for everyone else
Creating email responder

%cite each of the installed tools

BLAST, blastx, biopython, RNAfold, human genome,

\subsection{Implementation}




\subsubsection{BLAST}

Our project needed to be able 


\subsubsection{Server}

\subsubsection{Scheduler}


Once fasta file is retrieved, the scheduler spins up a new thread start the task. 
The thread is not a daemon, os it will live on.

\subsubsection{Email}




\section{Evaluation}

\subsection{Test Case}
We used a fasta file of five species of the Ebola virus as the focus

\subsection{Deliverables}
The project was not fully completed in time.

\bibliography{report}
\bibliographystyle{plain}

\end{document}