		\documentclass[10pt,letterpaper]{article}
\usepackage[latin1]{inputenc}
\usepackage{amsmath}
\usepackage{amsfonts}
\usepackage{amssymb}
\usepackage{graphicx}


\author{Andrew Rosen}
\title{MBD Talk 3 Writeup}
\begin{document}
	\maketitle
	This talk was about the global pandemic of HIV, which currently infects around 35 million people.
	The bible belt is a hot spot in the US.
	
	HIV is a retrovirus, named because it reverses the central dogma be replicated via reverse transcription, creating a single stranded DNA from some RNA.
	This is done by co-opting the cells of the host organism.
	The DNA of these cells are modified to become factories to create more HIV, which means infection is for life.
	
	HIV replication is impacted by dozens of miRNA, particular in regards to transcription \cite{klase2012micrornas}.
	Another interesting fact that was brought up  was that HIV exploits the exosome pathway \cite{nguyen2003evidence}.  
	If anything is blocked in this pathway, then HIV can't replicate.
	
	
	HIV has a fairly small genome but has a very complex behavior.
	Its complexity is due in part to it's ability to acquire the proteins it needs from the host \cite{guo1995hiv}.
	
	
	The presenter presented aa hypothesis that HIV is a ``Trojan Exosome.''
	This hypothesis explains why HIV exploits the Exosome pathways and why HIV acts like an exosome.
	It also gives explanation as to why HIV samples are contaminated with exosomes.
	This hypothesis predicts HIV vaccines actually help the virus, as the virus can identify and kill the T-Cells hunting for it.
	
	
	
	
	
		
		\bibliographystyle{plain}
		\bibliography{mbd}
	
	
\end{document}	
	
	HIV-1: subversive strategies in transmission and pathogenisis
	
	35 million infected currently.
	deaths caused by opportunistic infections
	global pandemic, but clustered in sub-saharan Africa
	Affects primarily people of color, large disparity. 
	Bible belt is the hot-spot for HIV  2013 CDC rates of diagonises of HIV infections
	
	Mechanisms
	When HIV infection occurs, hundred of genes are modulated, some by cellular/host response, but also mby the virus
	microRNA -HIV replication isimpacted by dozens of miRNA, most affect transcription  Klase Z et al 2012
	
	Exosomes- HIV exploits this pathway to  form HIV particles
	If you can block anything in this pathway, you stop HIV
	
	Activation of Endogenous Retroviruses occurs during HIV infection  
	Human Endogenous Retroviruses  (HERVS) can cause pathogenesis in humans
	
	HIV is a retrovirus
	attaches to the cell, has two receptors
	retro -Central dogma in reverse
	reverse transcription
	copied into genome, why infection is infection for life.
	
	HIV is fairly short but works with greater complexity
	Hypothesis:  HIV aquires host protiens from infected cless. Hildreth and Orentas Involement of a leukocyte adhesion
	Guo and Hildereth 1995 HIV 
	
	So how does HIV aquire protiens
	Lipid rafts are the sites for HIV budding
	Turned out to be the case for a lot of viruses.
	
	Modulaing cholesterol prevents infection
	removing cholesterol from membran of affected cell blocks release of infectious particles.  Particles are still emitted, but 
	
	20HpBCD as a vaginal microbicide
	Would kill HIV cells on contact
	20HpBCD as HIV Theraputic agent
	
	Subversion of Immune Recognition
	Is HIV biology linked to Exosome biogolgy
	
	Exosomes about same size as HIV
	Produced and taken up by many cell types
	Exosomes are a big deal in medicine now, being looked at for application to cancer
	
	Exosomes can be used to activate T-cells (Hwang et al) T cells exposed to exosomes from  immature  100:6670-6675
	
	Trojan Exosome Hypothesis:
	bc: HIV  exploits the exosome pathway
	HIV acts like an exosome
	preparations of HIV are always contaminated by exosomes
	Long story short it can idenify and kill the T-Cells hunting for it
	Exosomes predict vaccines will help the virus.
	
	


