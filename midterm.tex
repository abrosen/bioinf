\documentclass[10pt,a4paper]{article}
\usepackage[latin1]{inputenc}
\usepackage{amsmath}
\usepackage{amsfonts}
\usepackage{amssymb}
\usepackage{graphicx}
\author{Andrew Rosen}
\title{Midterm II}
\begin{document}
	\maketitle
	
	I decided to focus on the paper looking at anti-influenza aptamers  \cite{musafia2014designing}.
	% I might not understand it well, but I'll damn well \textbf{try}
	
	
	
	Musafia \textit{et al} examined a new way of fighting the influenza virus.
	Influenza is a particularly common viral disease that millions of people each year.
	It's particularly dangerous to adults over 65 \cite{centers2010estimates}.
	The primary mechanism of spreading  by coughing or sneezing \cite{stephenson2002epidemiology}.
	Expelled viral particles then bind to the victim's respiratory epithelium, a lining in the respiratory tract.
	
	Musafia \textit{et al} study using an aptamer to block the virus from attaching to the respiratory epithelium, inhibiting inflection.
	Aptamers are a type of molecule called oligonucleotides.
	These molecules are single-stranded DNA or RNA of approximately 15-45 nucleotides in length. 
	Aptamers  are highly discriminating and can be created to bind to very specific targets, while not binding to closely related ``targets.''\footnote{As far as I understand it.}
	
	The process of generating aptamers is called ``systematic evolution of ligands by exonential enrichment,'' or SELEX.
	This process generates a library of $ 10^{13} $ to $ 10 ^{15} $ random oligonucleotides, each 20-100 nucleotides long.
	The library is searched for sequences that can interact with the target; those few are chosen for the enrichment process.
	
	This paper examines a DNA aptamer called BV02 and compares it to generated aptamers of similar length and other properties.
	BV02 had been previously shown to be effective at preventing influenza infection by blocking the viral hemagglutinin \cite{jeon2004dna}.
	This aptamer was chosen by Musafia \textit{et al} as the starting point and base comparison to the other aptamers created by the authors.
		
	The method description involves a lengthy description of preparation  for biochemical experiments.  
	I won't pretend to understand the specifics, but this process describes how Musafia \textit{et al}  created their solutions and how a scientist\footnote{Unfortunate grad student.} could recreate it.
	
	
	The experiment itself was conducted on mice.
	The aptamers were applied to the mice premixed with the influenza virus.
	
	
	Their methods and models seem to be fairly effective.
	Method found better sequences
	Model is okay
	
	Might have been better to compare to aptamers that were closely related as well, simulating a slight mutation.
	
	
	
	\bibliography{bio}
	\bibliographystyle{plain}
\end{document}