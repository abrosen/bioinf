\documentclass[10pt,a4paper]{article}
\usepackage[latin1]{inputenc}
\usepackage{amsmath}
\usepackage{amsfonts}
\usepackage{amssymb}
\usepackage{graphicx}
\author{Andrew Rosen}
\title{Midterm II}
\begin{document}
	\maketitle
	
	I decided to focus on the paper looking at anti-influenza aptamers  \cite{musafia2014designing}.
	
	% I might not understand it well, but I'll damn well \textbf{try}
	
	\subsubsection*{Description}
	
	Musafia \textit{et al} examined a new way of fighting the influenza virus.
	Influenza is a particularly common viral disease that infects millions of people each year.
	It's particularly dangerous to adults over 65 \cite{centers2010estimates}.
	The primary mechanism of spreading  by coughing or sneezing \cite{stephenson2002epidemiology}.
	Expelled viral particles then bind to the victim's respiratory epithelium, a lining in the respiratory tract.
	
	Musafia \textit{et al} study the use of an aptamer to block the virus from attaching to the respiratory epithelium, inhibiting inflection.
	Aptamers are a type of molecule called oligonucleotides.
	These molecules are single-stranded DNA or RNA of approximately 15-45 nucleotides in length. 
	Aptamers  are highly discriminating and can be created to bind to very specific targets, while not binding to closely related ``targets.''\footnote{As far as I understand it.}
	
	The process of generating aptamers is called ``systematic evolution of ligands by exonential enrichment,'' or SELEX.
	This process generates a library of $ 10^{13} $ to $ 10 ^{15} $ random oligonucleotides, each 20-100 nucleotides long.
	The library is searched for sequences that can interact with the target; those few are chosen for the enrichment process.
	
	This paper examines a DNA aptamer called BV02 and compares it to generated aptamers of similar length and other properties.
	BV02 had been previously shown to be effective at preventing influenza infection by blocking the viral hemagglutinin \cite{jeon2004dna}.
	This aptamer was chosen by Musafia \textit{et al} as the starting point and base comparison to the other aptamers created by the authors.
		
	The method description involves a lengthy description of preparation  for biochemical experiments.  
	I won't pretend to understand the specifics, but this process describes how Musafia \textit{et al}  created their solutions and how a scientist\footnote{Unfortunate grad student.} could recreate it.
	The experiment to establish the efficacy of aptamers was conducted on mice.
	The aptamers were applied to the mice premixed with the influenza virus.
	
	This appears to have been done for a total of 96 aptamers. 
	Once the relative binding affinities \footnote{compared to BV02's ability to bind to viral hemagglutinin} were established, these aptamers were unceremoniously slapped into a database.
	Every fourth aptamer was noted as a member of the test set, and the remaining aptamers were used to create the training set.
	
	
	
	Various features were isolated which impacted the aptamers ability to bind to viral hemagglutinin.
	The effectiveness was impacted primarily by the aptamer's size, secondary structure, and the presence of repeated C nucleotides, among others.
	QSAR was used to create a mathematical model to predict the effectiveness of the apatamer's binding.
	
	\subsubsection*{Effectiveness}
	Musafia \textit{et al}'s results seem extremely promising.
	They found 14 aptamers that had 15 times the binding affinity of BV02 	 \cite{musafia2014designing}.
	They  were able to find the 6 most relevant descriptors which defined the effectiveness of the binding.
	Their algorithm was able to successfully predict the binding affinity of the aptamers.

	\subsubsection*{Relevence To Term Project}
	Not incredibly relevant.  
	Our project focuses on generating potential siRNA candidates for attacking a family of viruses.
	This  paper focuses on  using a machine learning algorithm to predict the effectiveness of an aptamer binding to influenza.
	
	
	\subsubsection*{Improvements}
	The first thing that raised my eyebrows  was that SELEX generated $ 10^{13 }$  to $ 10^{15} $ random sequences.
	That seems like an excessively large search space of random material, but this may be standard operating procedure for this field (again, I'm not a biologist).
	Since the user will be selecting which of these random sequences should be enriched, there's presumably some kind of scoring mechanism that can be used as a fitness function.
	
	Perhaps it would be possible to craft a genetic algorithm that does the same thing, but with a much smaller population than $ 10^{13} $ each generation.
	Select the best 100 sequences and apply  crossover and mutation we stop seeing improvement for a number of generations.
	
	\bibliography{bio}
	\bibliographystyle{plain}
	
	\subsubsection*{Additional comments}
	Under the ``Training and Test Data Sets for QSAR Study'' section, fourth is misspelled as ``forth.''
	
	%Also in doing some background research, I came across charts and figures that were created by scientists who a apparently world class biologists, yet somehow possessed of the need to use Comic Sans in their figures.
	%I have no explanation for this. 
\end{document}